%!TEX root = progress_report.tex

On the large scale, genuine users' activities on the Internet should be randomly distributed over time and space. There are certain areas and time periods associated with higher activity frequency, yet the transition between different time and space are usually smooth. Fraudulent activities, however, often comes with very high frequency within a certain time period, from a particular range of internet devices. For example, spam ratings on various review websites are usually published within a narrow time period, from a certain set of IP addresses. DDoS attacks often feature an overwhelming amount of packets such as TCP, UDP, and IMCP sent from a set of IP addresses (controlled by the attacker) to another particular set of IP addresses associated with the target site or service. Therefore, identifying dense subtensors in the activity databases would be helpful to detecting fraudulent or malicious activities.

The task of mining dense subtensors, however, is challenging in many aspects. The database could have a handful of dimensions, making brute force search impractical. Depending on particular application, the database could vary from easy to handle with a laptop to impossible for a regular cluster. Whatever method we want to use must be able to scale with the size of the problem. That's why SQL is a good platform for implementing the solution. The activity log databases are naturally relational databases easily described by tables. Also, SQL allows for easy and quick data retrieval and manipulation even on large datasets that don't fit in the computer memory, which frees us up from worrying about low-level problems such as data storage and querying.

In this project, we will be implementing the Dcube(\cite{shin2017d}) algorithm in SQL. We will use this subtensor mining algorithm to identify fraudulent ratings in review databases as well as malicious packets in the DARPA TCP dump database. We will then try to optimize our implementation using the indexing methods covered in this course. We will evaluate our implementation using various unit tests and on labeled data using standard tools such as the ROC curve. 